%%%%%%%%%%%%%%%%%%%%%%%%%%%%%%%%%%%%%%%%%
% Developer CV
% LaTeX Template
% Version 1.0 (28/1/19)
%
% This template originates from:
% http://www.LaTeXTemplates.com
%
% Authors:
% Jan Vorisek (jan@vorisek.me)
% Based on a template by Jan Küster (info@jankuester.com)
% Modified for LaTeX Templates by Vel (vel@LaTeXTemplates.com)
%
% License:
% The MIT License (see included LICENSE file)
%
%%%%%%%%%%%%%%%%%%%%%%%%%%%%%%%%%%%%%%%%%

%----------------------------------------------------------------------------------------
%	PACKAGES AND OTHER DOCUMENT CONFIGURATIONS
%----------------------------------------------------------------------------------------

\documentclass[9pt]{developercv} % Default font size, values from 8-12pt are recommended

%----------------------------------------------------------------------------------------

\begin{document}

%----------------------------------------------------------------------------------------
%	TITLE AND CONTACT INFORMATION
%----------------------------------------------------------------------------------------

\begin{minipage}[t]{0.45\textwidth} % 45% of the page width for name
	\vspace{-\baselineskip} % Required for vertically aligning minipages
	
	% If your name is very short, use just one of the lines below
	% If your name is very long, reduce the font size or make the minipage wider and reduce the others proportionately
	{\HUGE\textcolor{black}{\textbf{\MakeUppercase{Darshan}}}} % First name

	{\HUGE\textcolor{black}{\textbf{\MakeUppercase{Raul}}}} % Last name
	
	\vspace{6pt}
	
	
\end{minipage}
\begin{minipage}[t]{0.275\textwidth} % 27.5% of the page width for the first row of icons
	\vspace{-\baselineskip} % Required for vertically aligning minipages
	
	% The first parameter is the FontAwesome icon name, the second is the box size and the third is the text
	% Other icons can be found by referring to fontawesome.pdf (supplied with the template) and using the word after \fa in the command for the icon you want
	\icon{MapMarker}{9}{Mumbai}\\
	\icon{Phone}{9}{+91 9665781999}\\
	\icon{At}{9}{\href{mailto:darshan.raul@outlook.com}{darshan.raul@outlook.com}}\\	
\end{minipage}
\begin{minipage}[t]{0.275\textwidth} % 27.5% of the page width for the second row of icons
	\vspace{-\baselineskip} % Required for vertically aligning minipages
	
	% The first parameter is the FontAwesome icon name, the second is the box size and the third is the text
	% Other icons can be found by referring to fontawesome.pdf (supplied with the template) and using the word after \fa in the command for the icon you want
	\icon{Globe}{9}{\href{https://darshan-raul.github.io}{darshan-raul.github.io}}\\
	\icon{Github}{9}{\href{https://github.com/darshan-raul}{github.com/darshan-raul}}\\
	\icon{Wordpress}{9}{\href{https://cloudforte.wordpress.com}{cloudforte.wordpress.com}}\\
\end{minipage}

\vspace{0.5cm}

%----------------------------------------------------------------------------------------
%	INTRODUCTION, SKILLS AND TECHNOLOGIES
%----------------------------------------------------------------------------------------

\cvsect{summary}

\begin{minipage}[t]{1\textwidth} % 40% of the page width for the introduction text
	\vspace{-\baselineskip} % Required for vertically aligning minipages
	
 • Currently  serving as \textbf{Junior Devops Engineer} at FHS,Thane. \\ • \textbf{2.8+ years} of IT experience  \\
    • Around 2 years of experience in major AWS services with knowledge of some GCP,Azure services \\
    • Hands-on in services like IAM, EC2, S3, Lambda, Cloudwatch, RDS, Route53, VPC, Cloudformation.\\
    • Competent in Cloud Computing, Devops and Infrastructure as code concepts. \\
    • Passionate about automation. Created automation scripts in Python/Bash.\\
    • An effective team member with proven abilities to be a part of the team during the project phasea and training .\\
    • Possess excellent troubleshooting, interpersonal and time management skills.\\
    • Have created open source projects ranging from Angular web app, Containerized application, Serverless REST API.\\
    • Blog about the technologies and share experiences. \\
    

\end{minipage}


\cvsect{Skills}

\begin{tabular}{ |p{5cm}|p{6cm}|p{5cm}|p{3cm}|  }
 
 \hline
 \hspace{0.1cm}\textbf{Domain}& \hspace{0.1cm}\textbf{Hands-on}&\hspace{0.1cm}\textbf{Familiar with}\\
 \hline
\hspace{0.1cm} Cloud   & \hspace{0.1cm}Aws    &\hspace{0.1cm}Gcp,Azure\\
 \hline
 \hspace{0.1cm} OS&  \hspace{0.1cm}Ubuntu,Red Hat & \hspace{0.1cm}Amazon Linux  \\
 \hline
\hspace{0.1cm} Version control&  \hspace{0.1cm}Git,Github,Bitbucket,Codecommit & \hspace{0.1cm}Google repository,SVN  \\
 \hline
\hspace{0.1cm} Scripting/Programming &\hspace{0.1cm}Python,Shell,Angular,Javascript & \hspace{0.1cm}Typescript\\
 \hline
\hspace{0.1cm} CI/CD    &\hspace{0.1cm}Jenkins,CodePipeline & \hspace{0.1cm}GitlabCI,Buildbot\\
 \hline
\hspace{0.1cm} Configuration Mgmt/IAC &   \hspace{0.1cm}CloudFormation,Ansible,Puppet  & \hspace{0.1cm}Terraform,Chef\\
\hline
\hspace{0.1cm} Containers/Orchrestation& \hspace{0.1cm}Docker,docker Swarm  & \hspace{0.1cm}Kubernetes,ECS,GKE   \\
 \hline
 
 \hspace{0.1cm} Monitoring& \hspace{0.1cm}ELK stack, Graphana  & \hspace{0.1cm}Nagios,glances\\
 \hline
 \hspace{0.1cm} Miscellaneous& \hspace{0.1cm}MongoDb,MySQL,pytest,  & \hspace{0.1cm}Karma\\
 \hline
\end{tabular}
%----------------------------------------------------------------------------------------
%	EXPERIENCE
%----------------------------------------------------------------------------------------

\cvsect{Experience}

\begin{entrylist}
	\entry
		{Nov 2018 -\\Current}
		{Junior Devops Engineer}
		{Fountain Hill Systems,Thane}
		{\textbf{Environment:} AWS (EC2, S3, IAM, Cloudformation, Cloudwatch, Route53, RDS, EBS, EFS, Elasticache, ACM), ELK stack, Jenkins, Puppet, Nginx, Tomcat, Maven\\
		• Responsible for the deployment of architecture on 4 different (Prod, Stage, Qa, Dev) environments in 4 different AWS regions.\\
		• Deployment by building the jobs on Jenkins which would load artifact and Cloudformation template files to S3.\\
		• Deploying latest architecture by building and creating/updating Cloudformation Stacks.\\
		• Configuration management using the puppet scripts\\
		• Managing Load Balancing from ASG and doing RCA on Linux servers in case of failed deployment or sudden errors in the processing.\\
		• Monitored the AWS architectured using Cloudwatch and created alarms to notify anomalies.\\
		• Monitored the app logs using ELK stack\\
 • Assisted in updating the Cloudformation templates to better suit the latest dependencies.\\
 • Participated in project management through Github boards and JIRA.\\
 • Created documentation on various aspects of the project like security, configuration, improving efficiency and monitoring on Confluence.\\
  • Created POC for monitoring and auditing tools like Site 24/7,Manageengine,Cloudsecurity plus.\\
• Performed DR testing on the architecture to test for vulnerabilities. \\
• Exploring new open source tools and technologies.
 

\\ \texttt{AWS}\slashsep\texttt{Linux}\slashsep\texttt{Jenkins}\slashsep\texttt{Nginx}\slashsep\texttt{Tomcat}}
	\entry
		{Aug 2018\\\footnotesize{Part-time}}
		{Freelancer}
		{Saans Foundation,Pune}
		{• Created Static website on AWS S3: www.saans.org.in \\ 
	    • Freelancing and Volunteering for NGO. \\
	   • Created videos for social media and website\\\texttt{Html}\slashsep\texttt{Css}}
		
	\entry
		{Aug 2016}
		{Software Engineer}
		{Zensar Technologies,Pune}
		{• Worked with a team to manage a quickly growing messaging infrastructure in Hybrid Cloud and Onpremise's Windows and Linux server
environment ranging
from Email, Instant messaging, Journaling and BYOD mdm.\\• Provided daily monitoring, management, troubleshooting and issue resolution to
system and services in the Windows servers.\\ • Assisted in configuring a server infrastructure to move existing on-premise
environment to AWS cloud.\\ \texttt{Windows Server}\slashsep\texttt{AWS}\slashsep\texttt{Powershell}}
\end{entrylist}

%----------------------------------------------------------------------------------------
%	EDUCATION
%----------------------------------------------------------------------------------------

\cvsect{Education}

\begin{entrylist}
	\entry
		{2012 -- 2016}
		{BE Computer Science}
		{MES College of Engineering,Pune}
		
	\entry
		{2010 -- 2012}
		{HSC}
		{Nowrosjee Wadia College,Pune}
		
\end{entrylist}

%----------------------------------------------------------------------------------------
%	ADDITIONAL INFORMATION
%----------------------------------------------------------------------------------------

\cvsect{Projects}

\begin{minipage}[t]{1\textwidth} % 40% of the page width for the introduction text
	\vspace{-\baselineskip} % Required for vertically aligning minipages
	
  \textbf{AWS Konsole}:\\\\Angular 7 Web app to interact with AWS services • Deployed as static website on S3 • Backend is a Serverless API running on AWS Lambda • CI/CD pipeline using Jenkins\\\\ \icon{Github}{4}{\href{https://github.com/darshan-raul/awsdashboard}{github.com/darshan-raul/awsdashboard}}\\\\
\textbf{NativeScript Mobile App}:\\\\NativeScript/Angular 7 Mobile app to interact with AWS services • Can be deployed both as a IOS and Android App • Backend is a Serverless API running on AWS Lambda\\\\\icon{Github}{4}{\href{https://github.com/darshan-raul/NativeScript-Mobile-App}{github.com/darshan-raul/NativeScript-Mobile-App}}\\\\
    \textbf{Python AWS CLI}:\\\\Python based CLI application to interact with AWS services • More interactive way to deal with AWS operations rather than traditional CLI • Backend is a Serverless API running on AWS Lambda\\\\\icon{Github}{4}{\href{https://github.com/darshan-raul/Python-CLI-for-AWS}{github.com/darshan-raul/Python-CLI-for-AWS}}\\\\
        \textbf{ServerLess REST API}:\\\\Python Flask based API endpoints running on AWS Lambda • Lambda functions integrated with AWS API gateway • Monitoring the API responses/logs using AWS Cloudwatch and generating notifications to Slack on alarms.\\\\\icon{Github}{4}{\href{github.com/darshan-raul/SAM-Serverless-Application-Model-rest-api-lambda-}{github.com/darshan-raul/SAM-Serverless-Application-Model-rest-api-lambda-}}\\\\

\end{minipage}
%----------------------------------------------------------------------------------------

\end{document}
